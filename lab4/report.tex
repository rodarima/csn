\documentclass[a4paper]{article}
%\usepackage{amsfonts}
%\usepackage{amsmath}
%\usepackage{amsthm}
\usepackage[utf8]{inputenc}
%\usepackage{hyperref}
\usepackage{booktabs}
\usepackage{graphicx}
%\usepackage{subfig}
\usepackage[hyphens]{url}
\usepackage{hyperref}
\usepackage{siunitx}
%\sisetup{retain-zero-exponent=true}%
\usepackage{minted}
\newminted{py}{%
%		linenos,
		fontsize=\small,
		tabsize=2,
		mathescape,
}
\newminted{text}{%
%		linenos,
		fontsize=\small,
		tabsize=2,
		mathescape,
}

\title{Lab 4: Non-linear regression on dependency trees}
\author{Rodrigo Arias Mallo}
\date{\today}

\begin{document}
\maketitle

\section{Introduction}

A list of datasets with information of syntactic dependency trees are used in 
this report to derive conclusions about the relation of a metric and the size of 
the sentence $n$.

\subsection{Selection of the metric}
For selecting the metric, I built a small python program, that receives as input 
a list of elements, and a number. The program concatenate the list and the 
number by using \texttt{;} as separator, and a MD5 hash is computed over the 
resulting string. Finally, an element of the list is chosen by using the first 
byte of the hexadecimal hash, converted to a index modulus the size of the list.  
Then the item with the corresponding index is selected.

The last digit of my identification number is used as the input number, so the 
output is unique for me.
%
\begin{textcode}
% python choice.py 64718
degree 2nd moment
\end{textcode}
%
Finally the chosen metric is the degree 2nd moment $\langle k^2 \rangle$.

\section{Results}

All the analysis has been made in python, by using numeric and statistical 
packages. The datasets were read and the test
$$4-6/n \le \langle k^2 \rangle \le n-1$$
was failed in some elements, because the rounding errors. After allowing a small 
error $\epsilon = \num{5e-6}$ the test
$$4-6/n - \epsilon \le \langle k^2 \rangle \le n-1 + \epsilon$$
was succesfully passed in all languages.

\subsection{Summary}
In the table~\ref{tab:1} a summary of the properties of the degree sequences is 
shown. The sample mean and standard deviation of the metric $x$ are represented 
by $\overline x$ and $s_x$ respectively.
%
\begin{table}[h]
	\centering
	\begin{tabular}{lllll}
\toprule
 Language   & $N$   & $E$    & $\langle k \rangle$   & $\delta$   \\
\midrule
 Arabic     & 21531 & 68742  & \num{6.385E+00}             & \num{2.966E-04}  \\
 Basque     & 12207 & 25541  & \num{4.185E+00}             & \num{3.428E-04}  \\
 Catalan    & 36865 & 197075 & \num{1.069E+01}             & \num{2.900E-04}  \\
 Chinese    & 40298 & 180925 & \num{8.979E+00}             & \num{2.228E-04}  \\
 Czech      & 69303 & 257254 & \num{7.424E+00}             & \num{1.071E-04}  \\
 English    & 29634 & 193078 & \num{1.303E+01}             & \num{4.397E-04}  \\
 Greek      & 13283 & 43961  & \num{6.619E+00}             & \num{4.984E-04}  \\
 Hungarian  & 36126 & 106681 & \num{5.906E+00}             & \num{1.635E-04}  \\
 Italian    & 14726 & 55954  & \num{7.599E+00}             & \num{5.161E-04}  \\
 Turkish    & 20409 & 45625  & \num{4.471E+00}             & \num{2.191E-04}  \\
\bottomrule
\end{tabular}
	\caption{The measures of the datasets.}
	\label{tab:1}
\end{table}
%

\subsection{Models}

The models tested are presented in the table~\ref{tab:models}. The model 0 is 
used as reference, and has no parameters.
%
\begin{table}[h]
	\centering
	\begin{tabular}{cll}
		\toprule
		Model & Function & Parameters\\
		\midrule
		0  & $f(n) = (1-1/n)(5-6/n)$	& \\
		1  & $f(n) = (n/2)^b$					& $b$ \\
		2  & $f(n) = an^b$ 						& $a,b$\\
		3  & $f(n) = ae^{cn}$					& $a,c$\\
		4  & $f(n) = a\log n$					& $a$\\
		5  & $f(n) = an^be^{cn}$			& $a,b,c$\\
		1+ & $f(n) = (n/2)^b + d$			& $b,d$\\
		2+ & $f(n) = an^b + d$				& $a,b,d$\\
		3+ & $f(n) = ae^{cn} + d$			& $a,c,d$\\
		4+ & $f(n) = a\log n + d$			& $a,d$\\
		5+ & $f(n) = an^be^{cn} + d$	& $a,b,c,d$\\
		\bottomrule
	\end{tabular}
	\caption{The list of models to test.}
	\label{tab:models}
\end{table}
%

\subsection{Non-linear regression}
%
In the table~\ref{tab:AIC} the difference AIC metric is shown, with respect to 
the best model in each case. Note that the metric can be affected by the value 
of the outliers if we use the aggregate mean, so the full dataset is used to get 
the best parameters.
%
\begin{table}[H]
	\centering
	\begin{tabular}{rrrrrrr}
\toprule
   Language &     0 &     1 &    2 &    3 &   4 &   0+ \\
\midrule
          1 & \num{103.4} & \num{100.1} & \num{92.8} & \num{35.1} & nan &  \num{0.1} \\
          2 &  \num{79.0} &  \num{76.8} & \num{73.6} & \num{50.7} & nan & \num{37.3} \\
          3 &  \num{60.6} &  \num{58.8} & \num{56.4} & \num{42.2} & nan & \num{37.2} \\
\bottomrule
\end{tabular}
	\begin{tabular}{rrrrr}
\toprule
   Language &   1+ &   2+ &   3+ &   4+ \\
\midrule
          1 & \num{33.8} &  \num{0.0} & \num{11.1} &  nan \\
          2 & \num{55.2} &  \num{0.0} & \num{46.2} &  nan \\
          3 & \num{44.1} &  \num{0.0} & \num{40.3} &  nan \\
\bottomrule
\end{tabular}
	\caption{The $\Delta AIC$ of the models.}
	\label{tab:AIC}
\end{table}
%
We see that the model 5+ seems to better fit the data, followed by the 2+ model.

\section{Plots of the models}

The best model has been selected to be plotted along with the reference model 0, 
and the aggregate mean data points. The logarithmic scale is enabled in both 
axis.

%\begin{center}
\noindent
\includegraphics[width=.5\textwidth]{fig/Arabic-mean.png}
\includegraphics[width=.5\textwidth]{fig/Basque-mean.png}
\includegraphics[width=.5\textwidth]{fig/Catalan-mean.png}
\includegraphics[width=.5\textwidth]{fig/Chinese-mean.png}
\includegraphics[width=.5\textwidth]{fig/Czech-mean.png}
\includegraphics[width=.5\textwidth]{fig/English-mean.png}
\includegraphics[width=.5\textwidth]{fig/Greek-mean.png}
\includegraphics[width=.5\textwidth]{fig/Hungarian-mean.png}
\includegraphics[width=.5\textwidth]{fig/Italian-mean.png}
\includegraphics[width=.5\textwidth]{fig/Turkish-mean.png}
%\end{center}

\subsection{Model parameters}
The best model parameters are shown in the table~\ref{tab:params}. Each column 
is one parameter of the model, named $m\, p$ where $m$ is the model name and $p$ 
the parameter name.
%
\begin{table}[h]
	\centering
	\footnotesize
	\begin{tabular}{lrrrrrrrrr}
\toprule
 Language   &   1 b &   2 a &   2 b &   3 a &   3 c &   4 a &   5 a &   5 b &    5 c \\
\midrule
 Arabic     & \num{0.535} & \num{2.003} & \num{0.243} & \num{3.605} & \num{0.006} & \num{1.385} & \num{1.326} & \num{0.464} & \num{-0.010} \\
 Basque     & \num{0.767} & \num{1.887} & \num{0.340} & \num{3.236} & \num{0.022} & \num{1.813} & \num{1.106} & \num{0.741} & \num{-0.034} \\
 Catalan    & \num{0.604} & \num{2.730} & \num{0.191} & \num{4.229} & \num{0.006} & \num{1.581} & \num{1.875} & \num{0.376} & \num{-0.008} \\
 Chinese    & \num{0.932} & \num{1.267} & \num{0.526} & \num{2.208} & \num{0.059} & \num{1.876} & \num{0.734} & \num{1.159} & \num{-0.088} \\
 Czech      & \num{0.659} & \num{1.950} & \num{0.298} & \num{3.439} & \num{0.014} & \num{1.641} & \num{1.280} & \num{0.564} & \num{-0.017} \\
 English    & \num{0.635} & \num{2.954} & \num{0.179} & \num{4.396} & \num{0.006} & \num{1.658} & \num{2.132} & \num{0.345} & \num{-0.008} \\
 Greek      & \num{0.603} & \num{2.316} & \num{0.233} & \num{3.893} & \num{0.008} & \num{1.558} & \num{1.517} & \num{0.459} & \num{-0.011} \\
 Hungarian  & \num{0.712} & \num{2.503} & \num{0.285} & \num{4.582} & \num{0.011} & \num{1.996} & \num{1.782} & \num{0.464} & \num{-0.008} \\
 Italian    & \num{0.627} & \num{2.116} & \num{0.264} & \num{3.646} & \num{0.010} & \num{1.607} & \num{1.389} & \num{0.509} & \num{-0.013} \\
 Turkish    & \num{0.695} & \num{2.044} & \num{0.273} & \num{3.167} & \num{0.016} & \num{1.690} & \num{1.320} & \num{0.597} & \num{-0.025} \\
\bottomrule
\end{tabular}
	\begin{tabular}{lrrrrrrrr}
\toprule
 Language   &   1+ b &   1+ d &    2+ a &   2+ b &   2+ d &   3+ a &   3+ c &   3+ d \\
\midrule
 Arabic     &  \num{0.394} &  \num{1.609} &  \num{-7.046} & \num{-0.647} &  \num{5.493} & \num{-5.193} & \num{-0.179} &  \num{4.839} \\
 Basque     &  \num{0.550} &  \num{1.656} &  \num{-8.393} & \num{-0.616} &  \num{6.386} &  \num{5.000} &  \num{5.000} &  \num{5.000} \\
 Catalan    &  \num{0.372} &  \num{2.456} &  \num{-8.227} & \num{-0.591} &  \num{6.367} & \num{-4.571} & \num{-0.130} &  \num{5.434} \\
 Chinese    &  \num{0.737} &  \num{0.952} &  \num{-8.086} & \num{-0.572} &  \num{6.383} &  \num{5.000} &  \num{5.000} &  \num{5.000} \\
 Czech      &  \num{0.483} &  \num{1.666} &  \num{-7.956} & \num{-0.525} &  \num{6.506} & \num{-5.389} & \num{-0.169} &  \num{5.191} \\
 English    &  \num{0.372} &  \num{2.685} &  \num{-8.594} & \num{-0.664} &  \num{6.342} & \num{-4.741} & \num{-0.147} &  \num{5.536} \\
 Greek      &  \num{0.409} &  \num{2.042} &  \num{-7.831} & \num{-0.594} &  \num{6.157} & \num{-5.291} & \num{-0.162} &  \num{5.210} \\
 Hungarian  &  \num{0.510} &  \num{2.581} & \num{-11.970} & \num{-0.346} & \num{10.312} & \num{-5.949} & \num{-0.083} &  \num{7.418} \\
 Italian    &  \num{0.442} &  \num{1.832} &  \num{-7.905} & \num{-0.546} &  \num{6.370} & \num{-5.462} & \num{-0.170} &  \num{5.187} \\
 Turkish    &  \num{0.463} &  \num{1.680} &  \num{-7.846} & \num{-0.831} &  \num{5.251} &  \num{5.000} &  \num{5.000} &  \num{5.000} \\
\bottomrule
\end{tabular}
	\begin{tabular}{rrrr}
\toprule
   Language &    4+ a &   4+ d &   5+ a \\
\midrule
          1 & \num{-14.161} & \num{-7.465} & \num{-3.873} \\
          2 & \num{-14.161} & \num{-7.465} & \num{-3.873} \\
          3 & \num{-14.161} & \num{-7.465} & \num{-3.873} \\
\bottomrule
\end{tabular}
	\caption{The models parameters.}
	\label{tab:params}
\end{table}
%
\section{Methods}
%
% Methods should include any relevant methods not explained in this guide (for
% instance, decisions that you had to made and might have an influence on the
% results), initial values of the parameters used to call nls(), the techniques
% used to obtain those initial values and so on...
%
The proposed R function \texttt{nls} is the abreviature of Nonlinear 
Least-Squares. The basic procedure is to modify the parameters of the model in 
order to reduce the sum of the square distance between the data points and the 
predicted points of the model.
%
This function is available in the python package \texttt{scipy.optimize} as 
\texttt{least\_squares}. However the wrap function \texttt{curve\_fit} let us 
use it more easily, and is what has been used in order to fit the models.

The initial parameters are set by default to 1. Except in models 2+ and 3+ which 
had been modified a bit, but with no luck. The default parameters are defined in 
the model function:
%
\begin{pycode}
def m2d(n, a=0.06, b=0.8, d=1):   return a*n**b + d
def m3d(n, a=-1, c=-0.05, d=13):  return a*np.exp(c*n) + d
\end{pycode}
%
Some models can't find a optimum value and a exception is returned. In this case 
a second attempt is made, in order to find better parameters for the model. The 
function \texttt{differential\_evolution} let us use an evolutionary algorithm 
to find values where the least-squares fails. It takes a bit more, but almost 
always improves the parameters. For more details see \texttt{fit.py}.
%
\section{Discussion}
%
% The discussion should include a summary of the results and your
% interpretation. For instance, you should discuss
%
% - If there is a significance difference between the fit of the functions from 
%   null hypotheses and that of alternative hypotheses.
%
% - If the original data satisfy the assumption of homoscedasticity of the non-
%   linear regression methods considered here. In case that it does not hold,
%   you should explain how you have addressed it.
%
% - Discuss if the function giving the best fit gives a reasonably good fit
%   (e.g., checking visually that the best function provides a sufficiently good
%   fit).  Remember that the best function of an ensemble is not necessarily
%   the best in absolute terms.
%
% - The extent to which languages resemble or differ.
%
% The discussion section should also include some conclusions.
%
We see that the reference model 0 is not a very good predictor of the data. As 
$n$ increases the distance from the mean and the predicted value is bigger. The 
best model seems to be the model 5+, which is the one with the best AIC overall.
%
Homoscedasticity was not tested in any way, only visually looked.
%
By looking at all the models the selected one matches with the one that 
graphically appears to better fit the data.
%
We conclude that the mean metric can be successfully predicted by the nonlinear 
regression model 5+.

\end{document}
