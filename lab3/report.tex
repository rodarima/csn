\documentclass[a4paper]{article}
%\usepackage{amsfonts}
%\usepackage{amsmath}
%\usepackage{amsthm}
\usepackage[utf8]{inputenc}
%\usepackage{hyperref}
\usepackage{booktabs}
\usepackage{graphicx}
%\usepackage{subfig}
\usepackage[hyphens]{url}
\usepackage{hyperref}
\usepackage{minted}
\usepackage{siunitx}
%\sisetup{retain-zero-exponent=true}%
\newminted{py}{%
%		linenos,
		fontsize=\small,
		tabsize=2,
		mathescape,
}
\newminted{text}{%
%		linenos,
		fontsize=\small,
		tabsize=2,
		mathescape,
}

\title{Lab 3: Significance of network metrics}
\author{Rodrigo Arias Mallo}
\date{\today}

\begin{document}
\maketitle

\section{Introduction}
\subsection{The metric}

For selecting the metric, I set the seed to the last 5 digits of my DNI, as 
recently seen in some hash generation\footnote{	
\url{http://klondike.es/klog/2017/09/25/descifrando-las-bases-de-datos-del-referendum-catalan/}}, 
and ran the following command in python:
%
\begin{pycode}
from random import *

seed(64718)
metrics = ["clustering coefficient", "closeness centrality"]
r = randint(0, 1)

print(metrics[r])
\end{pycode}
%
Which produced the following output
%
\begin{textcode}
% python metric.py
clustering coefficient
\end{textcode}
%
So I decided to use clustering coefficient $C_{WS}$ as the metric $x$.


\subsection{Cleaning the data}

The datasets have been uncompressed in \texttt{data/}, and then processed by the 
script \texttt{prepare-data.sh}, which removes the header with the number of 
nodes and edges, and outputs the remaining edge list in the same file with a new 
extension \texttt{.edges}. The properties of the graphs can be seen at the 
table~\ref{tab:prop}.

\begin{table}[h]
	\centering
	\begin{tabular}{lrrrrr}
\toprule
 Language   &   $N$ &   $\overline n$ &   $s_n$ &   $\overline x$ &   $s_x$ \\
\midrule
 Arabic     &  4108 &          26.958 &  20.647 &           4.160 &   1.275 \\
 Basque     &  2933 &          11.335 &   6.527 &           4.143 &   1.089 \\
 Catalan    & 15053 &          25.572 &  13.618 &           4.962 &   0.824 \\
 Chinese    & 54238 &           6.249 &   3.310 &           3.218 &   1.066 \\
 Czech      & 25037 &          16.428 &  10.721 &           4.293 &   1.299 \\
 English    & 18779 &          24.046 &  11.223 &           5.170 &   0.802 \\
 Greek      &  2951 &          22.820 &  14.379 &           4.600 &   1.070 \\
 Hungarian  &  6424 &          21.660 &  12.565 &           5.956 &   1.706 \\
 Italian    &  4144 &          18.407 &  13.344 &           4.340 &   1.170 \\
 Turkish    &  6030 &          11.102 &   8.281 &           3.759 &   0.934 \\
\bottomrule
\end{tabular}
	\caption{Properties of the graphs after preproccesing.}
	\label{tab:prop}
\end{table}

\section{Experiments}

Based on some properties of a graph $G$, we want to create a sequence of random 
graphs $\langle G \rangle$ by using two different methods, and then test if they 
maintain a set of measures that we found on $G$.

The first model, the Erdös--Rényi graph takes as input the number of vertex 
$|V|$ and edges $|E|$ of $G$, and builds a new random graph with the same number 
of vertex and edges. The clustering coefficient is then computed for each graph 
in the sequence $\langle G_{ER} \rangle$ as $\langle x_{ER} \rangle$.
The switching model also produces a sequence of graphs $\langle G_S \rangle$ and 
the measurements $\langle x_S \rangle$.

\subsection{Erdös--Rényi model}

The ER model is implemented in \texttt{python} using the \texttt{networkx} 
package.  The \texttt{gnm\_random\_graph} creates a ER graph with parameters 
$|V|$ and $|E|$. An average of $T = 25$ graphs is performed. The measure is 
taken by calling \texttt{average\_clustering}.
%
\begin{table}[h]
	\centering
	\begin{tabular}{lrrr}
\toprule
 Language   &       $x$ &   $\overline x_{ER}$ &  $p(x_{ER} \ge x)$   \\
\midrule
 Arabic     & \num{1.885E-01} &            \num{2.958E-04} &           \num{0.000E+00} \\
 Basque     & \num{4.671E-02} &            \num{2.971E-04} &           \num{0.000E+00} \\
 Catalan    & \num{2.211E-01} &            \num{2.916E-04} &           \num{0.000E+00} \\
 Chinese    & \num{1.708E-01} &            \num{2.310E-04} &           \num{0.000E+00} \\
 Czech      & \num{1.217E-01} &            \num{1.124E-04} &           \num{0.000E+00} \\
 English    & \num{2.353E-01} &            \num{4.416E-04} &           \num{0.000E+00} \\
 Greek      & \num{1.338E-01} &            \num{4.800E-04} &           \num{0.000E+00} \\
 Hungarian  & \num{5.085E-02} &            \num{1.549E-04} &           \num{0.000E+00} \\
 Italian    & \num{1.437E-01} &            \num{4.868E-04} &           \num{0.000E+00} \\
 Turkish    & \num{2.236E-01} &            \num{2.317E-04} &           \num{0.000E+00} \\
\bottomrule
\end{tabular}

	\caption{The measures of ER model.}
	\label{tab:er}
\end{table}

We see that none of the generated graphs contain a value greater than the 
original one. We can conclude that, even the ER model keeps the number of nodes 
and edges, the clustering coefficient is smaller.

\subsection{Switching model}

This model is implemented in the R package BiRewire\cite{birewire} written by A.  
Gobbi, a fast implementation of the switching model \cite{gobbi14} 
\cite{gobbi17} with especial design for large graphs. The algorithm switches two 
edges while the degree distribution is kept. Given two edges $(u,v),(s,t)$, with 
all different vertex, the switches $(u,t), (s,v)$ and $(u,s), (v,t)$ are 
randomly performed, when no loops or multi-edges are introduced. A total of $T = 
25$ graphs are generated, starting with the graph for each language, and with a 
number of steps equal to $Q = |E|\log|E|$.  The clustering coefficient is 
computed by the function \texttt{transitivity(g, type='localaverage', 
isolates='zero')} and tabulated for each language in table~\ref{tab:switch}.
%
\begin{table}[h]
	\centering
	\begin{tabular}{lrrr}
\toprule
 Language   &   $x$ &   $\overline x_S$ &   $p(x_S \ge x)$ \\
\midrule
 Arabic     & 0.188 &             0.187 &            0.200 \\
 Basque     & 0.047 &             0.053 &            1.000 \\
 Catalan    & 0.221 &             0.145 &            0.000 \\
 Chinese    & 0.171 &             0.093 &            0.000 \\
 Czech      & 0.122 &             0.070 &            0.000 \\
 English    & 0.235 &             0.240 &            1.000 \\
 Greek      & 0.134 &             0.147 &            1.000 \\
 Hungarian  & 0.051 &             0.072 &            1.000 \\
 Italian    & 0.144 &             0.198 &            1.000 \\
 Turkish    & 0.224 &             0.247 &            1.000 \\
\bottomrule
\end{tabular}
	\caption{The measures of the switching model.}
	\label{tab:switch}
\end{table}
%
\section{Results}

In the ER model, the clustering coefficient is always smaller compared to the 
original graph. However, using the switching model the measures are similar, but 
sometimes still small.

\section{Discussion}

I don't know what is the effect of the model in the clustering coefficient, nor 
by what means can I learn that relation. Neither I understand the utility of 
especulating about such relation, at least without the ability of designing a 
posterior test of my hypothesis.

\section{Methods}

To avoid the low speed of the computation, I tested with some libraries that 
implement a efficient representation of the graph, and I found the BiRewire to 
be acceptable and simple. Also the \texttt{networkx} package has a very good 
documentation, and implements a lot of random generators, including the ER 
model.

I decided not to optimize anything without a good reason to do it\footnote{``The 
real problem is that programmers have spent far too much time worrying about 
efficiency in the wrong places and at the wrong times; premature optimization is 
the root of all evil (or at least most of it) in programming''-- Donald Knuth}; 
i.e.  after a thorough analysis. As the computation time was not too large, I 
avoided the selection of some advanced data structure to perform the 
computations.

\bibliographystyle{unsrt}
\bibliography{biblio}

\end{document}
