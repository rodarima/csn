\documentclass[a4paper]{article}
%\usepackage{amsfonts}
%\usepackage{amsmath}
%\usepackage{amsthm}
\usepackage[utf8]{inputenc}
%\usepackage{hyperref}
\usepackage{booktabs}
\usepackage{graphicx}
%\usepackage{subfig}
\usepackage[hyphens]{url}
\usepackage{hyperref}
\usepackage{minted}
\usepackage{siunitx}
%\sisetup{retain-zero-exponent=true}%
\newminted{py}{%
%		linenos,
		fontsize=\small,
		tabsize=2,
		mathescape,
}
\newminted{text}{%
%		linenos,
		fontsize=\small,
		tabsize=2,
		mathescape,
}

\title{Lab 3: Significance of network metrics}
\author{Rodrigo Arias Mallo}
\date{\today}

\begin{document}
\maketitle

\section{The metric}

For selecting the metric, I set the seed to the last 5 digits of my DNI, as 
recently seen in some hash generation\footnote{	
\url{http://klondike.es/klog/2017/09/25/descifrando-las-bases-de-datos-del-referendum-catalan/}}, 
and ran the following command in python:
%
\begin{pycode}
from random import *

seed(64718)
metrics = ["clustering coefficient", "closeness centrality"]
r = randint(0, 1)

print(metrics[r])
\end{pycode}
%
Which produced the following output
%
\begin{textcode}
% python metric.py
clustering coefficient
\end{textcode}
%
So I decided to use clustering coefficient $C_{WS}$ as a metric.

\section{Introduction}

\subsection{Cleaning the data}

The datasets have been uncompressed in \texttt{data/}, and then processed by the 
script \texttt{prepare-data.sh}, which removes the header with the number of 
nodes and edges, and outputs the remaining edge list in the same file with a new 
extension \texttt{.edges}. The properties of the graphs can be seen at the 
table~\ref{tab:prop}.

\begin{table}[h]
	\centering
	\begin{tabular}{lllll}
\toprule
 Language   & $N$   & $E$    & $\langle k \rangle$   & $\delta$   \\
\midrule
 Arabic     & 21531 & 68742  & \num{6.385E+00}             & \num{2.966E-04}  \\
 Basque     & 12207 & 25541  & \num{4.185E+00}             & \num{3.428E-04}  \\
 Catalan    & 36865 & 197075 & \num{1.069E+01}             & \num{2.900E-04}  \\
 Chinese    & 40298 & 180925 & \num{8.979E+00}             & \num{2.228E-04}  \\
 Czech      & 69303 & 257254 & \num{7.424E+00}             & \num{1.071E-04}  \\
 English    & 29634 & 193078 & \num{1.303E+01}             & \num{4.397E-04}  \\
 Greek      & 13283 & 43961  & \num{6.619E+00}             & \num{4.984E-04}  \\
 Hungarian  & 36126 & 106681 & \num{5.906E+00}             & \num{1.635E-04}  \\
 Italian    & 14726 & 55954  & \num{7.599E+00}             & \num{5.161E-04}  \\
 Turkish    & 20409 & 45625  & \num{4.471E+00}             & \num{2.191E-04}  \\
\bottomrule
\end{tabular}
	\caption{Properties of the graphs after preproccesing.}
	\label{tab:prop}
\end{table}

After some research I found the BiRewire R package \cite{birewire} written by A.  
Gobbi, a fast implementation of the switchin model \cite{gobbi14} 
\cite{gobbi17}.

\section{Proposition}

Based on some properties of a graph $\hat G$, we want to create a sequence of 
random graphs $\langle G \rangle$ by using two different methods, and then test 
if they maintain a set of measures that we found on $\hat G$.

The first model, the Erdös--Rényi graph takes as input the number of vertex 
$|V|$ and edges $|E|$ of $\hat G$, and builds a new random graph with the same 
number of vertex and edges. The clustering coefficient is then computed for each 
graph in the sequence $\langle G \rangle$ as $\langle X \rangle$.

We can consider the measurement $X$ as a random variable, with mean $E[X]$ and 
variance $VAR[X]$. By computing $T$ elements in the sequence, the sample mean 
$\overline X$ is an unbiased estimator of $E[X]$, and by the central limit 
theorem, the sample $\overline X$ is distributed 

\subsection{Erdös--Rényi model}

The ER model is implemented in \texttt{python} using the \texttt{networkx} 
package.  The \texttt{gnm\_random\_graph} creates a ER graph with parameters 
$|V|$ and $|E|$. An average of $T = 25$ graphs is performed. The measure is 
taken by calling \texttt{average\_clustering}.
%
\begin{table}[h]
	\centering
	\begin{tabular}{lrrr}
\toprule
 Language   &       $x$ &   $p(x_{ER} \ge x)$ &   $\overline x_{ER}$ \\
\midrule
 Arabic     & \num{1.885E-01} &           \num{0.000E+00} &            \num{2.958E-04} \\
 Basque     & \num{4.671E-02} &           \num{0.000E+00} &            \num{2.971E-04} \\
 Catalan    & \num{2.211E-01} &           \num{0.000E+00} &            \num{2.916E-04} \\
 Chinese    & \num{1.708E-01} &           \num{0.000E+00} &            \num{2.310E-04} \\
 Czech      & \num{1.217E-01} &           \num{0.000E+00} &            \num{1.124E-04} \\
 English    & \num{2.353E-01} &           \num{0.000E+00} &            \num{4.416E-04} \\
 Greek      & \num{1.338E-01} &           \num{0.000E+00} &            \num{4.800E-04} \\
 Hungarian  & \num{5.085E-02} &           \num{0.000E+00} &            \num{1.549E-04} \\
 Italian    & \num{1.437E-01} &           \num{0.000E+00} &            \num{4.868E-04} \\
 Turkish    & \num{2.236E-01} &           \num{0.000E+00} &            \num{2.317E-04} \\
\bottomrule
\end{tabular}
	\caption{The measures of ER model.}
	\label{tab:er}
\end{table}
%
We see that none of the generated graphs contain a value greater that the 
original one. We can conclude that, even the ER model keeps the number of nodes 
and edges, the clustering coefficient is smaller.

%
\begin{table}[h]
	\centering
	\begin{tabular}{lrrr}
\toprule
 Language   &   $x$ &   $\overline x_S$ &   $p(x_S \ge x)$ \\
\midrule
 Arabic     & 0.188 &             0.187 &            0.200 \\
 Basque     & 0.047 &             0.053 &            1.000 \\
 Catalan    & 0.221 &             0.145 &            0.000 \\
 Chinese    & 0.171 &             0.093 &            0.000 \\
 Czech      & 0.122 &             0.070 &            0.000 \\
 English    & 0.235 &             0.240 &            1.000 \\
 Greek      & 0.134 &             0.147 &            1.000 \\
 Hungarian  & 0.051 &             0.072 &            1.000 \\
 Italian    & 0.144 &             0.198 &            1.000 \\
 Turkish    & 0.224 &             0.247 &            1.000 \\
\bottomrule
\end{tabular}
	\caption{The measures of the switching model.}
	\label{tab:switch}
\end{table}
%
\section{Results}



\section{Discussion}

\section{Methods}



\bibliographystyle{unsrt}
\bibliography{biblio}

\end{document}
