% Saved into report.tex at 18:55 from the ShareLatex web

\documentclass{article}
\usepackage[utf8]{inputenc}


\usepackage[utf8]{inputenc}
\usepackage{graphicx}
\usepackage{listings}
\usepackage{color}

% Use better tables
\usepackage{booktabs}

% Units
\usepackage{siunitx}

\definecolor{dkgreen}{rgb}{0,0.6,0}
\definecolor{gray}{rgb}{0.5,0.5,0.5}
\definecolor{mauve}{rgb}{0.58,0,0.82}

\lstset{ %
  language=R,                     % the language of the code
  basicstyle=\footnotesize,       % the size of the fonts that are used for the code
  numbers=left,                   % where to put the line-numbers
  numberstyle=\tiny\color{gray},  % the style that is used for the line-numbers
  stepnumber=1,                   % the step between two line-numbers. If it's 1, each line
                                  % will be numbered
  numbersep=5pt,                  % how far the line-numbers are from the code
  backgroundcolor=\color{white},  % choose the background color. You must add \usepackage{color}
  showspaces=false,               % show spaces adding particular underscores
  showstringspaces=false,         % underline spaces within strings
  showtabs=false,                 % show tabs within strings adding particular underscores
  frame=single,                   % adds a frame around the code
  rulecolor=\color{black},        % if not set, the frame-color may be changed on line-breaks within not-black text (e.g. commens (green here))
  tabsize=2,                      % sets default tabsize to 2 spaces
  captionpos=b,                   % sets the caption-position to bottom
  breaklines=true,                % sets automatic line breaking
  breakatwhitespace=false,        % sets if automatic breaks should only happen at whitespace
  title=\lstname,                 % show the filename of files included with \lstinputlisting;
                                  % also try caption instead of title
  keywordstyle=\color{blue},      % keyword style
  commentstyle=\color{dkgreen},   % comment style
  stringstyle=\color{mauve},      % string literal style
  escapeinside={\%*}{*)},         % if you want to add a comment within your code
  morekeywords={*,...}            % if you want to add more keywords to the set
} 



\usepackage{listings}
\usepackage[dvipsnames]{xcolor}

\title{Lab 6 - CSN: Network dynamics}
\author{Pierre-Antoine Porte \\ \texttt{porte.pierreantoine@gmail.com}
\and Rodrigo Arias Mallo \\ \texttt{rodarima@gmail.com}}
\date{\today}

%%% \def\arraystretch{1.5}

\begin{document}

\maketitle

\section*{Introduction}

In this session, we had to model 3 different models following the dynamical 
principles of the Barabasi-Albert model. Those principles are: vertex growth and 
preferential attachment. The different models we needed to implements were the 
following:

\begin{itemize}
	\item Barabasi-Albert with the dynamical principles
	\item Barabasi-Albert with random attachment instead of preferential 
	attachment (only one dynamical principle).
	\item Barabasi-Albert with vertex growth suppressed (only one dynamical 
	principle).
\end{itemize}

Those models were simulated and the data kept in files so we could analyze 
mathematical properties of those models.

In this report we will show our results regarding those models and their 
analysis. We will discuss those results and explain them. Then we will explain 
how we implemented the model simulations and what we used to analyze it in the 
Methods section.

\section*{Results}
%
\begin{table}[h]
	\centering
	\begin{tabular}{lrrr}
\toprule
 Dataset        &      1 &      2 &      3 \\
\midrule
 $1_{\lambda}$  &  \num{1.593} &  \num{1.593} & \num{10.000} \\
 $2_{q}$        &  \num{0.500} &  \num{0.500} &  \num{0.100} \\
 $4_{\gamma}$   &  \num{2.189} &  \num{2.078} &  \num{2.000} \\
 $5_{\gamma}$   &  \num{2.000} &  \num{2.000} &  \num{2.000} \\
 $5_{k_{\max}}$ & \num{20.000} & \num{20.000} & \num{20.000} \\
\bottomrule
\end{tabular}
	\caption{Parameters used in each model.}
	\label{tab:param_dd}
\end{table}
\begin{table}[h]
	\centering
	\begin{tabular}{lrrr}
\toprule
 Dataset   &       A &       B &         C \\
\midrule
 $0 a$     &   \num{0.005} &   \num{0.002} &     \num{0.001} \\
 $1 a$     &   \num{0.434} &   \num{0.134} &     \num{0.110} \\
 $2 a$     &   \num{1.616} &   \num{3.348} &     \num{0.000} \\
 $2 b$     &   \num{0.350} &   \num{0.133} &     \num{1.166} \\
 $3 a$     &  \num{14.000} &   \num{8.637} &    \num{-0.000} \\
 $3 c$     &   \num{0.000} &   \num{0.000} &    \num{-0.527} \\
 $4 a$     &   \num{3.792} &   \num{1.222} &     \num{0.837} \\
 $4 d_1$   &  \num{-0.849} &   \num{1.117} &    \num{-0.871} \\
 $0+ a$    &   \num{0.003} &   \num{0.000} &     \num{0.002} \\
 $0+ d$    &  \num{14.000} &   \num{8.496} &    \num{-0.735} \\
 $1+ a$    &   \num{0.370} &   \num{0.072} &     \num{0.178} \\
 $1+ d$    &   \num{5.000} &   \num{5.000} &    \num{-5.000} \\
 $2+ a$    &   \num{0.500} &   \num{0.413} &     \num{0.000} \\
 $2+ b$    &   \num{0.466} &   \num{0.300} &     \num{1.155} \\
 $2+ d$    &   \num{5.000} &   \num{5.000} &    \num{-0.058} \\
 $3+ a$    &   \num{0.289} &   \num{0.289} &     \num{0.289} \\
 $3+ c$    &   \num{0.853} &   \num{0.853} &     \num{0.853} \\
 $3+ d$    & \num{-10.000} & \num{-10.000} &   \num{-10.000} \\
 $4+ a$    &  \num{12.971} &   \num{1.230} &    \num{48.860} \\
 $4+ d_1$  & \num{952.161} &   \num{1.413} & \num{27177.319} \\
 $4+ d_2$  & \num{-80.698} &  \num{-0.060} &  \num{-500.000} \\
\bottomrule
\end{tabular}
	\caption{Parameters used with the models at $t = 1$.}
	\label{tab:param_dt1}
\end{table}
%

\section*{Discussion}

As explained in section Methods, we used $m_0$ and $n_0$ always with the same 
values (which can be however different for each model). We never compared the 
models with different $m_0$ or $n_0$ for the same model. It could have indicate 
us if the model behave differently given different graph in input. We could have 
went further and do this but we preferred to focus on the analysis of our 
different models.

\section*{Methods} \label{methods}

For the default \textit{Barabasi-Albert} and model with random attachment the 
initial graph was an empty graph with only one vertex.

For the model without vertex growth, we used an unconnected graph with $t_max$ 
vertices. Because we have no vertex growth, the vertices are not increasing and 
$n_0$ = $n_tmax$.

For the three models we used $m_0 = 0$ to used "\textit{clean}" and 
"\textit{empty}" graphs.

We measured the growth of the vertex degree over time and the degree 
distribution for each model. The vertex degree was measured over the time for $t 
= 1, 10, 100, 1000, 10000$ successively.

We used python for generating the models, to store the results and to analyze 
the data. For each BA model $M$ a folder in \texttt{data/model$M$/} contains all 
the results produced from this model. Inside, the degree sequence is stored in 
the file \texttt{dseq.txt}, the degree distribution in \texttt{dd.txt} and for 
each $T$ in the arrival time, we produced \texttt{dt\_$T$.txt} tracing the 
degree of the vertex arriving at time $T$.


\end{document}
